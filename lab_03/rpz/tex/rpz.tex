%% Преамбула TeX-файла

% 1. Стиль и язык
\documentclass[utf8, 12pt]{G7-32} % Стиль (по умолчанию будет 14pt)

% Остальные стандартные настройки убраны в preamble.inc.tex.
\sloppy

% Настройки стиля ГОСТ 7-32
% Для начала определяем, хотим мы или нет, чтобы рисунки и таблицы нумеровались в пределах раздела, или нам нужна сквозная нумерация.
\EqInChapter % формулы будут нумероваться в пределах раздела
\TableInChapter % таблицы будут нумероваться в пределах раздела
\PicInChapter % рисунки будут нумероваться в пределах раздела
\usepackage{slashbox}

% Добавляем гипертекстовое оглавление в PDF
\usepackage[
bookmarks=true, colorlinks=true, unicode=true,
urlcolor=black,linkcolor=black, anchorcolor=black,
citecolor=black, menucolor=black, filecolor=black,
]{hyperref}

% Изменение начертания шрифта --- после чего выглядит таймсоподобно.
% apt-get install scalable-cyrfonts-tex

\IfFileExists{cyrtimes.sty}
    {
        \usepackage{cyrtimespatched}
    }
    {
        % А если Times нету, то будет CM...
    }

\usepackage{graphicx}   % Пакет для включения рисунков

% С такими оно полями оно работает по-умолчанию:
% \RequirePackage[left=20mm,right=10mm,top=20mm,bottom=20mm,headsep=0pt]{geometry}
% Если вас тошнит от поля в 10мм --- увеличивайте до 20-ти, ну и про переплёт не забывайте:
\geometry{right=20mm}
\geometry{left=30mm}


% Пакет Tikz
\usepackage{tikz}
\usetikzlibrary{arrows,positioning,shadows}

% Произвольная нумерация списков.
\usepackage{enumerate}

% ячейки в несколько строчек
\usepackage{multirow}

% itemize внутри tabular
\usepackage{paralist,array}

% Центрирование подписей к плавающим окружениям
\usepackage[justification=centering]{caption}

% объявляем новую команду для переноса строки внутри ячейки таблицы
\newcommand{\specialcell}[2][c]{%
	\begin{tabular}[#1]{@{}c@{}}#2\end{tabular}}


\usepackage{amsmath}

% Настройки листингов.
\ifPDFTeX
% Листинги

\usepackage{listings}
\usepackage{wrapfig}
\usepackage{xcolor}


\definecolor{lispgreen}{RGB}{154, 228, 151}
\definecolor{lightgray}{gray}{0.97}
\definecolor{violet}{rgb}{0.8, 0, 0.7}

\lstdefinestyle{lispStyle}{
        language=Lisp,
        basicstyle=\ttfamily\small,
        numbers=left,
        numberstyle=\tiny,
        keywordsprefix=&,
        keywordstyle=\color{lispgreen!70!black},
        stringstyle=\color{violet},
        showstringspaces=false,
        backgroundcolor=\color{lightgray},
        frame=single}


% Значения по умолчанию
\lstset{
  basicstyle= \footnotesize,
  breakatwhitespace=true,% разрыв строк только на whitespacce
  breaklines=true,       % переносить длинные строки
%   captionpos=b,          % подписи снизу -- вроде не надо
  inputencoding=koi8-r,
  numbers=left,          % нумерация слева
  numberstyle=\footnotesize,
  showspaces=false,      % показывать пробелы подчеркиваниями -- идиотизм 70-х годов
  showstringspaces=false,
  showtabs=false,        % и табы тоже
  stepnumber=1,
  tabsize=4,              % кому нужны табы по 8 символов?
  frame=single,
  escapeinside={(*}{*)}, %выделение
  literate={а}{{\selectfont\char224}}1
  {б}{{\selectfont\char225}}1
  {в}{{\selectfont\char226}}1
  {г}{{\selectfont\char227}}1
  {д}{{\selectfont\char228}}1
  {е}{{\selectfont\char229}}1
  {ё}{{\"e}}1
  {ж}{{\selectfont\char230}}1
  {з}{{\selectfont\char231}}1
  {и}{{\selectfont\char232}}1
  {й}{{\selectfont\char233}}1
  {к}{{\selectfont\char234}}1
  {л}{{\selectfont\char235}}1
  {м}{{\selectfont\char236}}1
  {н}{{\selectfont\char237}}1
  {о}{{\selectfont\char238}}1
  {п}{{\selectfont\char239}}1
  {р}{{\selectfont\char240}}1
  {с}{{\selectfont\char241}}1
  {т}{{\selectfont\char242}}1
  {у}{{\selectfont\char243}}1
  {ф}{{\selectfont\char244}}1
  {х}{{\selectfont\char245}}1
  {ц}{{\selectfont\char246}}1
  {ч}{{\selectfont\char247}}1
  {ш}{{\selectfont\char248}}1
  {щ}{{\selectfont\char249}}1
  {ъ}{{\selectfont\char250}}1
  {ы}{{\selectfont\char251}}1
  {ь}{{\selectfont\char252}}1
  {э}{{\selectfont\char253}}1
  {ю}{{\selectfont\char254}}1
  {я}{{\selectfont\char255}}1
  {А}{{\selectfont\char192}}1
  {Б}{{\selectfont\char193}}1
  {В}{{\selectfont\char194}}1
  {Г}{{\selectfont\char195}}1
  {Д}{{\selectfont\char196}}1
  {Е}{{\selectfont\char197}}1
  {Ё}{{\"E}}1
  {Ж}{{\selectfont\char198}}1
  {З}{{\selectfont\char199}}1
  {И}{{\selectfont\char200}}1
  {Й}{{\selectfont\char201}}1
  {К}{{\selectfont\char202}}1
  {Л}{{\selectfont\char203}}1
  {М}{{\selectfont\char204}}1
  {Н}{{\selectfont\char205}}1
  {О}{{\selectfont\char206}}1
  {П}{{\selectfont\char207}}1
  {Р}{{\selectfont\char208}}1
  {С}{{\selectfont\char209}}1
  {Т}{{\selectfont\char210}}1
  {У}{{\selectfont\char211}}1
  {Ф}{{\selectfont\char212}}1
  {Х}{{\selectfont\char213}}1
  {Ц}{{\selectfont\char214}}1
  {Ч}{{\selectfont\char215}}1
  {Ш}{{\selectfont\char216}}1
  {Щ}{{\selectfont\char217}}1
  {Ъ}{{\selectfont\char218}}1
  {Ы}{{\selectfont\char219}}1
  {Ь}{{\selectfont\char220}}1
  {Э}{{\selectfont\char221}}1
  {Ю}{{\selectfont\char222}}1
  {Я}{{\selectfont\char223}}1
}

% Стиль для псевдокода: строчки обычно короткие, поэтому размер шрифта побольше
\lstdefinestyle{pseudocode}{
  basicstyle=\small,
  keywordstyle=\color{black}\bfseries\underbar,
  language=Pseudocode,
  numberstyle=\footnotesize,
  commentstyle=\footnotesize\it
}

% Стиль для обычного кода: маленький шрифт
\lstdefinestyle{realcode}{
  basicstyle=\scriptsize,
  numberstyle=\footnotesize
}

% Стиль для коротких кусков обычного кода: средний шрифт
\lstdefinestyle{simplecode}{
  basicstyle=\footnotesize,
  numberstyle=\footnotesize
}

% Стиль для BNF
\lstdefinestyle{grammar}{
  basicstyle=\footnotesize,
  numberstyle=\footnotesize,
  stringstyle=\bfseries\ttfamily,
  language=BNF
}

% Определим свой язык для написания псевдокодов на основе Python
\lstdefinelanguage[]{Pseudocode}[]{Python}{
  morekeywords={each,empty,wait,do},% ключевые слова добавлять сюда
  morecomment=[s]{\{}{\}},% комменты {а-ля Pascal} смотрятся нагляднее
  literate=% а сюда добавлять операторы, которые хотите отображать как мат. символы
    {->}{\ensuremath{$\rightarrow$}~}2%
    {<-}{\ensuremath{$\leftarrow$}~}2%
    {:=}{\ensuremath{$\leftarrow$}~}2%
    {<--}{\ensuremath{$\Longleftarrow$}~}2%
}[keywords,comments]

% Свой язык для задания грамматик в BNF
\lstdefinelanguage[]{BNF}[]{}{
  morekeywords={},
  morecomment=[s]{@}{@},
  morestring=[b]",%
  literate=%
    {->}{\ensuremath{$\rightarrow$}~}2%
    {*}{\ensuremath{$^*$}~}2%
    {+}{\ensuremath{$^+$}~}2%
    {|}{\ensuremath{$|$}~}2%
}[keywords,comments,strings]

% Подписи к листингам на русском языке.
\renewcommand\lstlistingname{\cyr\CYRL\cyri\cyrs\cyrt\cyri\cyrn\cyrg}
\renewcommand\lstlistlistingname{\cyr\CYRL\cyri\cyrs\cyrt\cyri\cyrn\cyrg\cyri}

\else
\usepackage{local-minted}
\fi

% Полезные макросы листингов.
% Любимые команды
\newcommand{\Code}[1]{\textbf{#1}}


\begin{document}

\frontmatter % выключает нумерацию ВСЕГО; здесь начинаются ненумерованные главы: реферат, введение, глоссарий, сокращения и прочее.

% Команды \breakingbeforechapters и \nonbreakingbeforechapters
% управляют разрывом страницы перед главами.
% По-умолчанию страница разрывается.

% \nobreakingbeforechapters
% \breakingbeforechapters

% Также можно использовать \Referat, как в оригинале
%\begin{abstract}
%	Титульный лист. Эта страница нужна мне, чтобы не сбивалась нумерация страниц
%	\cite{Dh}
%	\cite{Bayer}
%	\cite{Habr1}
%	\cite{Noise_func}
%	\cite{Ulich}

%Это пример каркаса расчётно-пояснительной записки, желательный к использованию в РПЗ проекта по курсу РСОИ.

%Данный опус, как и более новые версии этого документа, можно взять по адресу (\url{https://github.com/rominf/latex-g7-32}).

%\end{abstract}
% НАЧАЛО ТИТУЛЬНОГО ЛИСТА
\begin{center}
	\hfill \break
	\textit{
		\normalsize{Государственное образовательное учреждение высшего профессионального образования}}\\ 
	
	\textit{
		\normalsize  {\bf  «Московский государственный технический университет}\\ 
		\normalsize  {\bf имени Н. Э. Баумана»}\\
		\normalsize  {\bf (МГТУ им. Н.Э. Баумана)}\\
	}
	\noindent\rule{\textwidth}{2pt}
	\hfill \break
	\noindent
	\makebox[0pt][l]{ФАКУЛЬТЕТ}%
	\makebox[\textwidth][c]{«Информатика и системы управления»}%
	\\
	\noindent
	\makebox[0pt][l]{КАФЕДРА}%
	\makebox[\textwidth][r]{«Программное обеспечение ЭВМ и информационные технологии»}%
	\\
	\hfill\break
	\hfill \break
	\hfill \break
	\hfill \break
	\normalsize{\bf Отчет}\\
	\normalsize{\bf По лабораторной работе №2}\\
	\hfill \break
	\large{По курсу «Функциональное и логическое программирование»}\\
	\hfill \break
	\hfill \break
	\hfill \break
	\hfill \break	
        \begin{flushright}
            Студент: Киселев А.М.
        \end{flushright}
        \begin{flushright}
            Группа: ИУ7-66
        \end{flushright}
        \begin{flushright}
            Преподаватель: Толпинская Н.Б.
        \end{flushright}
	\hfill \break
	\hfill \break
	\hfill \break
\end{center}
\hfill \break
\hfill \break
\begin{center} Москва 2019\end{center}

\thispagestyle{empty} % 
% КОНЕЦ ТИТУЛЬНОГО ЛИСТА


%%% Local Variables: 
%%% mode: latex
%%% TeX-master: "rpz"
%%% End: 


\tableofcontents

%\include{10-defines}
%\include{11-abbrev}

%\Introduction



\mainmatter % это включает нумерацию глав и секций в документе ниже

\chapter{ Выполнение работы}
\label{cha:analysis}

\section{ Первое задание}

Написать предикат, который принимает два числа-аргумента и возвращает T, если первое число не меньше второго.

Данная функция представлена в \ref{lst:res1}

\begin{lstlisting}[style=lispStyle, caption={ Предикат, который проводит сравнение чисел a и b.},
                    label={lst:res1}]
(defun f (a b) (>= a b))
\end{lstlisting}

\section{ Второе задание}

Переписать функцию how-alike, приведенную в лекции и использующую COND, используя конструкции IF, AND/OR.

\section{ Третье задание}

Чем принципиально отличаются функции cons, list, append?

Пусть (setf lst1 `(a b))(setf lst2 `(c d))

Каковы результаты следующих выражений? Сами выражения и их результаты представленны в \ref{lst:task3}

Принципиальные отличия cons и list заключаются в:
\begin{enumerate}
    \item количестве аргументов(cons принимает только два аргумента, а list - неограниченное количество;
    \item в конечном результате. Cons возвращает точечную пару, которая в итоге может оказаться списком. List - только список.
\end{enumerate}

Принципиальное отличие append от list:
\begin{enumerate}
    \item в аргументах, подающихся на вход. Append принимает на вход только списки. List может принимать как списки, так и атомы;
\end{enumerate}

\begin{lstlisting}[style=lispStyle, caption={ Выражения и их результата.},
                    label={lst:task3}]
(cons lst1 lst2)
;((a b) c d)
(list lst1 lst2)
;((a b) (c d))
(append lst1 lst2)
;(a b c d)
\end{lstlisting}

\section{ Четвертое задание}

Каковы результаты вычисления следующих выражений?

Сами выражения и их результаты представлены в \ref{lst:res4}

\begin{lstlisting}[style=lispStyle, caption={ Выражения и их результата.},
                    label={lst:res4}]
(reverse ())
;Nil
(last ())
;Nil
(reverse `(a))
;(a)
(last `(a))
;(a)
(reverse `((a b c)))
;((a b c))
(last `((a b c)))
;((a b c))
\end{lstlisting}

\section{ Пятое задание}

Написать, по крайней мере, два варианта функции, которая возвращает последний элемент своего списка-аргумента.

Функции представлены в \ref{lst:res5}

\begin{lstlisting}[style=lispStyle, caption={ Функции, возвращающие последний элемент списка},
                    label={lst:res5}]
(defun f1(lst)
    (if (null lst)
        lst
        (or (f1 (cdr lst))
            (if (null (cdr lst))
                lst
            )
        )
    )
)

(defun f2(lst)
    (car (reverse lst))
)
\end{lstlisting}

\section{ Шестое задание}

Написать, по крайней мере, два варианта функции, которая возвращает свой список-аргумент без последнего элемента.

Функции представлены в \ref{lst:res6}

\begin{lstlisting}[style=lispStyle, caption={ Функции, возвращающие список без последнего элемента},
                    label={lst:res6}]
(defun f1(lst)
        (if (null lst)
            lst
            (if (not (null (cdr lst)))
                (cons (car lst) (f1 (cdr lst)))
            )
        )
)

(defun f2(lst)
    (reverse (cdr (reverse lst)))
)
\end{lstlisting}

\section{ Седьмое задание}
Написать простой вариант игры в кости, в котором бросаются две правильные кости. Если сумма выпавших очков равна 7 или 11 -- выиграш, если выпало (1, 1) или (6, 6) -- игрок получает право снова бросить кости, во всех остальных случаях ход переходит ко второму игроку, но запоминается сумма выпавших очков. Если второй игрок не выиграывает абсолютно, то выигрывает тот игрок, у которого больше очков. Результат игры и значения выпавших костей выводить на экран с помощью функции print.

Листинг программы представлен в \ref{lst:res7}

\begin{lstlisting}[style=lispStyle, caption={ Программа симмуляции игры в кости.},
                    label={lst:res7}]

(setf *random-state* (make-random-state t))

(defun throw-dice()
    (list (+ 1 (random 6)) (+ 1 (random 6)))
)

(defun player-throw(name)
    (setf thrw (throw-dice))
    (print name)
    (princ ": ")
    (write thrw)
    (if (or (equal thrw `(1 1))
        (equal thrw `(6 6))
        )
        (player-throw name)
    )
    thrw
)

(defun dice-sum(thrw)
    (+ (car thrw) (cadr thrw))
)

(defun abs-win(sum)
    (if (or (eq sum 7) (eq sum 11))
        T
        Nil
    )
)

(defun game()
    (setf sum1 (dice-sum (player-throw `Player1)))
    (setf res `(Player1 won!))
    
    (if (not (abs-win sum1))
        (progn   
            (setf sum2 (dice-sum (player-throw `Player2)))
            (if (eq sum1 sum2)
                (game)
                (if (> sum2 sum1)
                    (setf res `(Player2 won!))
                )
            )
        )
        ()
    )
    res
)
\end{lstlisting}

%\chapter{ Ответы на теоритические вопросы}
\label{cha:design}
\section{ Как синтаксически представляется программа на Lisp, и как она хранится в памяти?}
    Программа представляется в виде s-выражений, программа так же хранится в пямяти вместе с данными.
\section{ Как трактуются элементы списка?}
    Элемент списка трактуются как списковая ячейка. У нее есть указатель на данные и на хвост списка.
\section{ Порядок реализации программы.}
    Метод обработки определяется так как устроена техника(императивно, т.е. последовательно)

\begin{lstlisting}
вызов s-выражения
    подготовка arg1
    применяется функция
    возвращаяется arg1

    ...

применить функцию | первый элемент s-выражения к аргументам
результат
\end{lstlisting}

%\chapter{ Технологический раздел}
\label{cha:design}

%\chapter{Исследовательский раздел}



\backmatter %% Здесь заканчивается нумерованная часть документа и начинаются ссылки и
            %% заключение

%\Conclusion % заключение к отчёту

В результате выполнения крусового проекта были получены следующие основные результаты:


%% % Список литературы при помощи BibTeX
% Юзать так:
%
% pdflatex rpz
% bibtex rpz
% pdflatex rpz

\bibliographystyle{gost780u}
\bibliography{rpz}


%%% Local Variables: 
%%% mode: latex
%%% TeX-master: "rpz"
%%% End: 


%\appendix   % Тут идут приложения

%%\chapter{Картинки}
%\label{cha:appendix1}

%\begin{figure}
%\centering
%\caption{Картинка в приложении. Страшная и ужасная.}
%\end{figure}

%%% Local Variables: 
%%% mode: latex
%%% TeX-master: "rpz"
%%% End: 

%%\chapter{Еще картинки}
%\label{cha:appendix2}

%\begin{figure}
%\centering
%\caption{Еще одна картинка, ничем не лучше предыдущей. Но %надо же как-то заполнить место.}
%\end{figure}

%%% Local Variables: 
%%% mode: latex
%%% TeX-master: "rpz"
%%% End: 


\end{document}

%%% Local Variables:
%%% mode: latex
%%% TeX-master: t
%%% End:
