\chapter{ Выполнение работы}
\label{cha:analysis}

\section{ Первое задание}
Представить следующие списки в виде списочных выражений:

\section{ Второе задание}
Используя только функции CAR и CDR, написать выражения, возвращающие второй, третий, четвуртый элементы заданного списка.

Получение отдельных элементов списка (1 2 3 4 5 6) представлено в \ref{lst:cadr}
\begin{lstlisting}[style=lispStyle, caption={Получение элементов списка с помощью команд car и cdr.},
                    label={lst:cadr}]
(cadr `(1 2 3 4 5 6))   ; получить второй элемент
(caddr `(1 2 3 4 5 6))   ; получить третий элемент
(cadddr `(1 2 3 4 5 6))   ; получить четвертый элемент
\end{lstlisting}

\section{ Третье задание}
Что будет в результате вычисления выражений?
Сами выражения и результаты их вычислений представлены в \ref{lst:thirdtask}
\begin{lstlisting}[style=lispStyle, caption={Получение элементов списка с помощью команд car и cdr.},
                    label={lst:thirdtask}]
;a)
(caadr `((blue cube)(red pyramid)))
;red

;b)
(cdar `((abc)(def)(ghi)))
;Nil

;c)
(cadr `((abc)(def)(ghi)))
;(def)

;d)
(caddr `((abc)(def)(ghi)))
;(ghi)
\end{lstlisting}

\section{ Четвертое задание}
Напишите результат вычисления выражений. Выполнение задания представлено в \ref{lst:4task}
\begin{lstlisting}[style=lispStyle, caption={Выражения и их результат},
                label={lst:4task}]
(list 'Fred 'and Wilma)
;Ошибка

(list 'Fred'(and Wilma))
;(A (B C))

(cons Nil Nil)
(Nil)

(cons T Nil)
(T)

(cons Nil T)
(Nil . T)

(list Nil)
(Nil)

(cons (T)Nil)
;Ошибка, T - воспринимается как функция

(list '(one two)'(free temp))
;((one two) (free temp))


(cons 'Fred'(and Wilma))
;(Fred and Wilma)

(cons 'Fred'(Wilma))
;(Fred Wilma)

(list Nil Nil)
;(Nil Nil)

(list T Nil)
;(T Nil)

(cons T(list Nil))
;(T Nil)

(list (T)Nil)
;Ошибка, функция T неопределена

(cons `(one two)'(free temp))
;((one two) free temp)
\end{lstlisting}

\section{ Пятое задание}
Написать функцию (f ar1 ar2 ar3 ar4), возвращающую список \ref{lst:func}
\begin{equation}
\label{lst:func}
((ar1 ar2)(ar3 ar4))
\end{equation}
