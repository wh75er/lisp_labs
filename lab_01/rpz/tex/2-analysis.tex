\chapter{ Выполнение работы}
\label{cha:analysis}

\section{ Первое задание}
Представить следующие списки в виде списочных выражений:

\section{ Второе задание}
Используя только функции CAR и CDR, написать выражения, возвращающие второй, третий, четвуртый элементы заданного списка.

Получение отдельных элементов списка (1 2 3 4 5 6) представлено в \ref{lst:cadr}
\begin{lstlisting}[style=lispStyle, caption={Получение элементов списка с помощью команд car и cdr.},
                    label={lst:cadr}]
(cadr `(1 2 3 4 5 6))   ; получить второй элемент
(caddr `(1 2 3 4 5 6))   ; получить третий элемент
(cadddr `(1 2 3 4 5 6))   ; получить четвертый элемент
\end{lstlisting}

\section{ Третье задание}
Что будет в результате вычисления выражений?
Сами выражения и результаты их вычислений представлены в \ref{lst:thirdtask}
\begin{lstlisting}[style=lispStyle, caption={Получение элементов списка с помощью команд car и cdr.},
                    label={lst:thirdtask}]
;a)
(caadr `((blue cube)(red pyramid)))
;red

;b)
(cdar `((abc)(def)(ghi)))
;Nil

;c)
(cadr `((abc)(def)(ghi)))
;(def)

;d)
(caddr `((abc)(def)(ghi)))
;(ghi)
\end{lstlisting}
