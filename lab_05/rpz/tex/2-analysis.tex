\chapter{ Выполнение работы}
\label{cha:analysis}

В \ref{lst:defabc} определены a b и c

\begin{lstlisting}[style=lispStyle, caption={ Определения a b и c},
                    label={lst:defabc}]
(setf a 2)
(setf b 3)
(setf c 5)
\end{lstlisting}

В \ref{lst:res1} представлены s-выражения и их результат

\begin{lstlisting}[style=lispStyle, caption={ Выражения и их результат},
                    label={lst:res1}]
a
;2

b
;3

c
;5

`a
;a

`(+ a c)
(+ a c)

(a)
;Ошибка

(eval `a)
;2
\end{lstlisting}

Добавление функций в \ref{lst:deffuncs}


\begin{lstlisting}[style=lispStyle, caption={ Определение функций a и b},
                    label={lst:deffuncs}]
(defun a() `b)
(defun b() 4)
\end{lstlisting}

Выражения и результаты их вычисления после определения функций в \ref{lst:deffuncs} представлены в \ref{lst:res2}


\begin{lstlisting}[style=lispStyle, caption={ Выражения и их результаты},
                    label={lst:res2}]
a
;2

b
;3

c
;5

`a
;a

`(+ a c)
;(+ a c)

(eval `a)
;2

(a)
;b

(eval `a)
;2

(+ a a)
;4

(+ (b) b)
;7

(b b b)
;Ошибка
\end{lstlisting}

Добавление условия к вышеперечисленным( после \ref{lst:deffuncs}) представлено в \ref{lst:redefa}

\begin{lstlisting}[style=lispStyle, caption={ Переопределение a},
                    label={lst:redefa}]
(setf a b)
\end{lstlisting}

Выражения и их результаты после переопределения a в \ref{lst:redefa} представлены в \ref{lst:res3}

\begin{lstlisting}[style=lispStyle, caption={ Переопределение a},
                    label={lst:res3}]
a
;3

b
;3

c
;5

`a
a
`(+ a c)
(+ a c)

(a)
;b

(eval `a)
;3
\end{lstlisting}

Дальнейшие переопределения и выражения с результатами представлены в \ref{lst:res4}

\begin{lstlisting}[style=lispStyle, caption={ Результаты вычислений},
                    label={lst:res4}]
;Добавление переопределения c
(setf c b)

a
;3

b
;3

c
;3

`a
;a

`(+ a c)
;(+ a c)

(a)
;b

(eval `a)
;3


;Добавление функции
(defun a(x y)(+ x y))

a
;3

b
;3

c
;3

`a
;a

`(+ a c)
;(+ a c)

(a)
;Ошибка

(eval `a)
;3

(a a a)
;6 

(a b a)
;6


;Добавление функции
(defun b(x y)(setf b (+ b 1))( * x y b))

a
;3

b
;3

(b b b)
;36
\end{lstlisting}

Функция, которая вычисляет катет по гипотенузе и другому катету представлена в \ref{lst:triangle}

\begin{lstlisting}[style=lispStyle, caption={ Функция, которая вычисляет катет по гипотенузе и другому катету},
                    label={lst:triangle}]
(defun f(c a) (sqrt (- ( * c c) ( * a a))))
\end{lstlisting}
