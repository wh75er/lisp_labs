\chapter{ Ответы на теоритические вопросы}
\label{cha:design}
\section{ Как синтаксически представляется программа на Lisp, и как она хранится в памяти?}
    Программа представляется в виде s-выражений, программа так же хранится в пямяти вместе с данными.
\section{ Как трактуются элементы списка?}
    Элемент списка трактуются как списковая ячейка. У нее есть указатель на данные и на хвост списка.
\section{ Порядок реализации программы.}
    Метод обработки определяется так как устроена техника(императивно, т.е. последовательно)

\begin{lstlisting}
вызов s-выражения
    подготовка arg1
    применяется функция
    возвращаяется arg1

    ...

применить функцию | первый элемент s-выражения к аргументам
результат
\end{lstlisting}
