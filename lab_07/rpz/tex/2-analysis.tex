\chapter{ Выполнение работы}
\label{cha:analysis}

\section{ Первое задание}

Написать предикат, который принимает два числа-аргумента и возвращает T, если первое число не меньше второго.

Данная функция представлена в \ref{lst:res1}

\begin{lstlisting}[style=lispStyle, caption={ Предикат, который проводит сравнение чисел a и b.},
                    label={lst:res1}]
(defun f (a b) (>= a b))
\end{lstlisting}

\section{ Второе задание}

Переписать функцию how-alike, приведенную в лекции и использующую COND, используя конструкции IF, AND/OR.

\section{ Третье задание}

Чем принципиально отличаются функции cons, list, append?

Пусть (setf lst1 `(a b))(setf lst2 `(c d))

Каковы результаты следующих выражений? Сами выражения и их результаты представленны в \ref{lst:task3}

Принципиальные отличия cons и list заключаются в:
\begin{enumerate}
    \item количестве аргументов(cons принимает только два аргумента, а list - неограниченное количество;
    \item в конечном результате. Cons возвращает точечную пару, которая в итоге может оказаться списком. List - только список.
\end{enumerate}

Принципиальное отличие append от list:
\begin{enumerate}
    \item в аргументах, подающихся на вход. Append принимает на вход только списки. List может принимать как списки, так и атомы;
\end{enumerate}

\begin{lstlisting}[style=lispStyle, caption={ Выражения и их результата.},
                    label={lst:task3}]
(cons lst1 lst2)
;((a b) c d)
(list lst1 lst2)
;((a b) (c d))
(append lst1 lst2)
;(a b c d)
\end{lstlisting}

\section{ Четвертое задание}

Каковы результаты вычисления следующих выражений?

Сами выражения и их результаты представлены в \ref{lst:res4}

\begin{lstlisting}[style=lispStyle, caption={ Выражения и их результата.},
                    label={lst:res4}]
(reverse ())
;Nil
(last ())
;Nil
(reverse `(a))
;(a)
(last `(a))
;(a)
(reverse `((a b c)))
;((a b c))
(last `((a b c)))
;((a b c))
\end{lstlisting}

\section{ Пятое задание}

Написать, по крайней мере, два варианта функции, которая возвращает последний элемент своего списка-аргумента.

Функции представлены в \ref{lst:res5}

\begin{lstlisting}[style=lispStyle, caption={ Функции, возвращающие последний элемент списка},
                    label={lst:res5}]
(defun f1(lst)
    (if (null lst)
        lst
        (or (f1 (cdr lst))
            (if (null (cdr lst))
                lst
            )
        )
    )
)

(defun f2(lst)
    (car (reverse lst))
)
\end{lstlisting}

\section{ Шестое задание}

Написать, по крайней мере, два варианта функции, которая возвращает свой список-аргумент без последнего элемента.

Функции представлены в \ref{lst:res6}

\begin{lstlisting}[style=lispStyle, caption={ Функции, возвращающие список без последнего элемента},
                    label={lst:res6}]
(defun f1(lst)
        (if (null lst)
            lst
            (if (not (null (cdr lst)))
                (cons (car lst) (f1 (cdr lst)))
            )
        )
)

(defun f2(lst)
    (reverse (cdr (reverse lst)))
)
\end{lstlisting}

\section{ Седьмое задание}
Написать простой вариант игры в кости, в котором бросаются две правильные кости. Если сумма выпавших очков равна 7 или 11 -- выиграш, если выпало (1, 1) или (6, 6) -- игрок получает право снова бросить кости, во всех остальных случаях ход переходит ко второму игроку, но запоминается сумма выпавших очков. Если второй игрок не выиграывает абсолютно, то выигрывает тот игрок, у которого больше очков. Результат игры и значения выпавших костей выводить на экран с помощью функции print.

Листинг программы представлен в \ref{lst:res7}

\begin{lstlisting}[style=lispStyle, caption={ Программа симмуляции игры в кости.},
                    label={lst:res7}]

(setf *random-state* (make-random-state t))

(defun throw-dice()
    (list (+ 1 (random 6)) (+ 1 (random 6)))
)

(defun player-throw(name)
    (setf thrw (throw-dice))
    (print name)
    (princ ": ")
    (write thrw)
    (if (or (equal thrw `(1 1))
        (equal thrw `(6 6))
        )
        (player-throw name)
    )
    thrw
)

(defun dice-sum(thrw)
    (+ (car thrw) (cadr thrw))
)

(defun abs-win(sum)
    (if (or (eq sum 7) (eq sum 11))
        T
        Nil
    )
)

(defun game()
    (setf sum1 (dice-sum (player-throw `Player1)))
    (setf res `(Player1 won!))
    
    (if (not (abs-win sum1))
        (progn   
            (setf sum2 (dice-sum (player-throw `Player2)))
            (if (eq sum1 sum2)
                (game)
                (if (> sum2 sum1)
                    (setf res `(Player2 won!))
                )
            )
        )
        ()
    )
    res
)
\end{lstlisting}
