\chapter{ Выполнение работы}
\label{cha:analysis}

\section{ Первое задание}
    Написать функцию, которая переводит температуру в системе Фаренгейта в температуру по Цельсию.

Данная функция представлена в виде lymbda-выражения и вызывается с помощью apply. Результаты представлены в  \ref{lst:res1}

\begin{lstlisting}[style=lispStyle, caption={ Функция и результат работы с аргументом 451.},
                    label={lst:res1}]
(apply #'(lambda (temp) (float ( * (/ 5 9) (- temp 32)))) `(451))
;232.77777
\end{lstlisting}

\section{ Второе задание}
Что получится при вычислении каждого из выражений?

Выражения и их результаты представлены в \ref{lst:res2}

\begin{lstlisting}[style=lispStyle, caption={ Выражения и их результаты.},
                    label={lst:res2}]
(list 'cons t NIL)
;(cons t nil)

(eval (eval (list 'cons t NIL)))
;Ошибка

(apply #'cons `(t NIL))
;(t)

(list `eval NIL)
;(eval NIL)

(eval (list 'cons t NIL))
;(t)

(eval NIL)
;nil

(eval (list `eval NIL))
;(nil)
\end{lstlisting}

\section{ Третье задание}
Написать функцию, которая принимает целое число и возвращает первое четное число, не меньшее аргумента.

Функция представлена в \ref{lst:res3}

\begin{lstlisting}[style=lispStyle, caption={ Представление функции.},
                    label={lst:res3}]
(defun evenr(x) (if (evenp x) (eval x) (+ x 1)))
\end{lstlisting}

\section{ Четвертое задание}
Написать функцию, которая принимает число и возвращает число того же знака, но с модулем на 1 больше модуля аргумента.

Функция представлена в \ref{lst:res4}

\begin{lstlisting}[style=lispStyle, caption={ Представление функции.},
                    label={lst:res4}]
(defun abs-plus(x) (if (>= x 0) (+ (abs x) 1) ( * -1 (+ (abs x) 1))))
\end{lstlisting}

\section{ Пятое задание}
Написать функцию, которая принимает два числа и возвращает список этих чисел, расположенный по возрастанию.

Функция представлена в \ref{lst:res5}

\begin{lstlisting}[style=lispStyle, caption={ Представление функции.},
                    label={lst:res5}]
(defun greater-list(a b) (if (>= b a) (list a b) (list b a)))
\end{lstlisting}

\section{ Шестой задание}
Написать функцию, которая принимает три числа и возвращает T только тогда, когда первое число расположено между вторым и третьим.

Функция представлена в \ref{lst:res6}

\begin{lstlisting}[style=lispStyle, caption={ Представление функции.},
                    label={lst:res6}]
(defun num-between(b a c) (if (or (and (>= b a) (<= b c)) (and (>= b c) (<= b a))) (eval T) ()))
\end{lstlisting}

\section{ Седьмое задание}
Каков результат вычисления следующих выражений?

Выражения и их результаты представлены в \ref{lst:res7}

\begin{lstlisting}[style=lispStyle, caption={ Выражения и их результаты.},
                    label={lst:res7}]
(and `fee `fie `foe)
;foe

(or `fee `fie `foe)
;fee

(and (equal `abc `abc) `yes)
;yes

(or nil `fie `foe)
;fie

(and nil `fie `foe)
;nil

(or (equal `abc `abc) `yes)
;t
\end{lstlisting}
