\chapter{ Выполнение работы}
\label{cha:analysis}

\section{ Первое задание}

Выражения и их результат представлены в \ref{lst:result}

\begin{lstlisting}[style=lispStyle, caption={ Выражения и их результат},
                    label={lst:result}]
(equal 3 (abs -3))
;T

(equal (+ 1 2) 3)
;T

(equal ( * 4 7) 21)
;Nil

(equal ( * 2 3)(+ 7 2))
;Nil

(equal (- 7 3)( * 3 2))
;Nil

(equal (abs (- 2 4)) 3))
;Nil
\end{lstlisting}

Диаграмма вычисления представленных выражений приведено в \ref{lst:diagram}

\begin{lstlisting}[style=lispStyle, caption={ Представление диаграммы вычисления выражений в виде деревьев},
                    label={lst:diagram}]
(equal 3 (abs -3))
запуск обработки функции equal
    3 вычисляется как 3
    запус обработкии функции abs
        -3 вычесляется как -3
    применяется equal к 3 и 3
    возвращается T

(equal (+ 1 2) 3)
запуск обработки функции equal
    запуск обработки функции +
        1 вычисляется как 1
        2 вычисляется как 2
        возвращается 3
    3 вычисляется как 3
    применяется equal к 3 и 3
    возвращается T

(equal ( * 4 7) 21)
запуск обработки функции equal
    запуск обработки функции *
        4 вычисляется как 4
        7 вычисляется как 7
        возвращается 24
    21 вычисляется как 21
    применяется equal к 24 и 21
    возвращается Nil

(equal ( * 2 3)(+ 7 2))
запуск обработки функции equal
    запуск обработки функции *
        2 вычисляется как 2
        3 вычисляется как 3
        возвращается 6
    запуск обработки функции +
        7 вычисляется как 7
        2 вычисляется как 2
        возвращается 9
    применяется equal к 6 и 9
    возвращается Nil

(equal (- 7 3)( * 3 2))
запуск обработки функции equal
    запуск обработки функции -
        7 вычисляется как 7
        3 вычисляется как 3
        возвращается 4
    запуск обработки функции *
        3 вычисляется как 3
        2 вычисляется как 2
        возвращается 6
    применяется equal к 4 и 6
    возвращается Nil

(equal (abs (- 2 4)) 3)
запуск обработки функции equal
    запуск обработки функции abs
        запуск обработки функции -
            2 вычисляется как 2
            4 вычисляется как 4
            возвращается -2
        примененяется abs к -2
        возвращается 2
    3 вычисляется как 3
    применяется equal к 2 и 3
    возвращается Nil
\end{lstlisting}
